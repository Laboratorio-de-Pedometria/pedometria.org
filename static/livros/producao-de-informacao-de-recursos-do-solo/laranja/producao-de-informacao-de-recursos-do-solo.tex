% Options for packages loaded elsewhere
\PassOptionsToPackage{unicode}{hyperref}
\PassOptionsToPackage{hyphens}{url}
\PassOptionsToPackage{dvipsnames,svgnames*,x11names*}{xcolor}
%
\documentclass[
  11pt,
  a4paper,
  dvipsnames]{tufte-book}
\usepackage{lmodern}
\usepackage{amssymb,amsmath}
\usepackage{ifxetex,ifluatex}
\ifnum 0\ifxetex 1\fi\ifluatex 1\fi=0 % if pdftex
  \usepackage[T1]{fontenc}
  \usepackage[utf8]{inputenc}
  \usepackage{textcomp} % provide euro and other symbols
\else % if luatex or xetex
  \usepackage{unicode-math}
  \defaultfontfeatures{Scale=MatchLowercase}
  \defaultfontfeatures[\rmfamily]{Ligatures=TeX,Scale=1}
\fi
% Use upquote if available, for straight quotes in verbatim environments
\IfFileExists{upquote.sty}{\usepackage{upquote}}{}
\IfFileExists{microtype.sty}{% use microtype if available
  \usepackage[]{microtype}
  \UseMicrotypeSet[protrusion]{basicmath} % disable protrusion for tt fonts
}{}
\makeatletter
\@ifundefined{KOMAClassName}{% if non-KOMA class
  \IfFileExists{parskip.sty}{%
    \usepackage{parskip}
  }{% else
    \setlength{\parindent}{0pt}
    \setlength{\parskip}{6pt plus 2pt minus 1pt}}
}{% if KOMA class
  \KOMAoptions{parskip=half}}
\makeatother
\usepackage{xcolor}
\IfFileExists{xurl.sty}{\usepackage{xurl}}{} % add URL line breaks if available
\IfFileExists{bookmark.sty}{\usepackage{bookmark}}{\usepackage{hyperref}}
\hypersetup{
  pdftitle={Produção de Informação de Recursos do Solo},
  pdfauthor={Alessandro Samuel-Rosa},
  colorlinks=true,
  linkcolor=Maroon,
  filecolor=Maroon,
  citecolor=Blue,
  urlcolor=Blue,
  pdfcreator={LaTeX via pandoc}}
\urlstyle{same} % disable monospaced font for URLs
\usepackage{longtable,booktabs}
% Correct order of tables after \paragraph or \subparagraph
\usepackage{etoolbox}
\makeatletter
\patchcmd\longtable{\par}{\if@noskipsec\mbox{}\fi\par}{}{}
\makeatother
% Allow footnotes in longtable head/foot
\IfFileExists{footnotehyper.sty}{\usepackage{footnotehyper}}{\usepackage{footnote}}
\makesavenoteenv{longtable}
\usepackage{graphicx,grffile}
\makeatletter
\def\maxwidth{\ifdim\Gin@nat@width>\linewidth\linewidth\else\Gin@nat@width\fi}
\def\maxheight{\ifdim\Gin@nat@height>\textheight\textheight\else\Gin@nat@height\fi}
\makeatother
% Scale images if necessary, so that they will not overflow the page
% margins by default, and it is still possible to overwrite the defaults
% using explicit options in \includegraphics[width, height, ...]{}
\setkeys{Gin}{width=\maxwidth,height=\maxheight,keepaspectratio}
% Set default figure placement to htbp
\makeatletter
\def\fps@figure{htbp}
\makeatother
\setlength{\emergencystretch}{3em} % prevent overfull lines
\providecommand{\tightlist}{%
  \setlength{\itemsep}{0pt}\setlength{\parskip}{0pt}}
\setcounter{secnumdepth}{5}
\usepackage{booktabs}
\usepackage[utf8]{inputenc}
\usepackage[T1]{fontenc}
\usepackage[brazilian]{babel}
% https://www.ctan.org/pkg/fancyhdr
\usepackage{fancyhdr}
\pagestyle{fancy}
\fancyhead[RO]{\textsl{\nouppercase{\rightmark}}}
\fancyhead[LE]{\textsl{\nouppercase{\leftmark}}}
\fancyfoot[LE, RO]{\thepage}
\renewcommand{\headrulewidth}{0.4pt}
% http://tug.ctan.org/tex-archive/macros/latex/contrib/tufte-latex/
\publisher{Universidade Tecnológica Federal do Paraná}
% https://tex.stackexchange.com/questions/443018/package-inputenc-error-unicode-char-%CC%81-u301inputenc
% \DeclareUnicodeCharacter{0301}{*************************************}
% \DeclareUnicodeCharacter{2265}{AAAAAAAAAAAAAAAAAAAAAAAAAAAAAAAAAAAAA}

% https://tex.stackexchange.com/questions/473859/subtitle-in-tufte-book
\makeatletter
\newcommand{\plainsubtitle}{}%     plain-text-only subtitle
\newcommand{\subtitle}[1]{%
  \gdef\@subtitle{#1}%
  \renewcommand{\plainsubtitle}{#1}% use provided plain-text title
  \ifthenelse{\isundefined{\hypersetup}}%
    {}% hyperref is not loaded; do nothing
    {\hypersetup{pdftitle={\plaintitle: \plainsubtitle{}}}}% set the PDF metadata title
}
\renewcommand{\maketitlepage}[0]{%
  \cleardoublepage%
  {%
  \sffamily%
  \begin{fullwidth}%
    \fontsize{18}{20}\selectfont\par\noindent\textcolor{darkgray}{\allcaps{\thanklessauthor}}%
    \vspace{11.5pc}%
    \fontsize{36}{40}\selectfont\par\noindent\textcolor{darkgray}{\allcaps{\thanklesstitle}}%
    \vspace{5pc}%
    \fontsize{24}{28}\selectfont\par\noindent\textcolor{darkgray}{\allcaps{\plainsubtitle}}%
    \vfill%
    \fontsize{14}{16}\selectfont\par\noindent\allcaps{\thanklesspublisher}%
  \end{fullwidth}%
  }
  \thispagestyle{empty}%
  \clearpage%
}
\makeatother

% https://tex.stackexchange.com/questions/77999/remove-indent-of-paragraph-and-add-line-skip-with-tufte-latex/
\makeatletter
% Paragraph indentation and separation for normal text
\renewcommand{\@tufte@reset@par}{%
  \setlength{\RaggedRightParindent}{0.0pc}%
  \setlength{\JustifyingParindent}{0.0pc}%
  \setlength{\parindent}{0pc}%
  % \setlength{\parskip}{\baselineskip}%
}
\@tufte@reset@par


\pagecolor{BurntOrange}
\usepackage[]{natbib}
\bibliographystyle{apalike}

\title{Produção de Informação de Recursos do Solo}
\usepackage{etoolbox}
\makeatletter
\providecommand{\subtitle}[1]{% add subtitle to \maketitle
  \apptocmd{\@title}{\par {\large #1 \par}}{}{}
}
\makeatother
\subtitle{Suporte ao processo de ensino e aprendizagem}
\author{Alessandro Samuel-Rosa}
\date{2020-12-01}

\begin{document}
\maketitle

\newpage
\nopagecolor
\begin{fullwidth}
~\vfill
\thispagestyle{empty}
\setlength{\parindent}{0pt}
\setlength{\parskip}{\baselineskip}

\par{\smallcaps{Publicado pela Universidade Tecnológica Federal do Paraná, Câmpus Santa Helena}}

\par{Curso de Agronomia/Laboratório de Pedometria\\Prolongamento da Rua Cerejeira, s/n\\Bairro São Luiz\\CEP 85892-000\\Santa Helena - PR - Brasil\\Telefone Geral +55 (45) 3268-8800}

\par\smallcaps{www.pedometria.org}

\par\includegraphics[width=1.5cm]{img/cc-by-nc-sa.png} Exceto quando proveniente de outras fontes ou onde especificado o contrário, o conteúdo deste documento está licenciado sob uma licença \href{http://creativecommons.org/licenses/by-nc-sa/4.0/}{Creative Commons Atribuição-NãoComercial-CompartilhaIgual 4.0 Internacional}.

\par\textit{Última edição em \today.}
\end{fullwidth}

{
\hypersetup{linkcolor=}
\setcounter{tocdepth}{1}
\tableofcontents
}
\hypertarget{prefuxe1cio}{%
\chapter*{Prefácio}\label{prefuxe1cio}}
\addcontentsline{toc}{chapter}{Prefácio}

\hypertarget{introduuxe7uxe3o}{%
\chapter{Introdução}\label{introduuxe7uxe3o}}

\hypertarget{semana-01}{%
\chapter{Semana 01}\label{semana-01}}

Início das atividades

\hypertarget{encontro}{%
\section{Encontro}\label{encontro}}

Planejamento de atividades (50 min)

O encontro desta semana será nosso primeiro encontro desde março, quando as aulas presenciais foram suspensas por razão da pandemia do COVID‑19. Além de saber como você está, o objetivo é revisar o planejamento de atividades para o período não-presencial que vai até dezembro. Nós conversaremos sobre as leituras semanais que você terá que fazer de trechos do \href{https://biblioteca.ibge.gov.br/visualizacao/livros/liv95017.pdf}{Manual Técnico de Pedologia} e do \href{https://ainfo.cnptia.embrapa.br/digital/bitstream/item/199517/1/SiBCS-2018-ISBN-9788570358004.pdf}{Sistema Brasileiro de Classificação de Solos}. Também conversaremos sobre as avaliações individuais, que envolvem questionários e fóruns semanais. Por último, vamos tratar do trabalho em grupo, que consiste na participação em um projeto de extensão chamado \href{https://pt.wikipedia.org/wiki/Wikip\%C3\%A9dia:Outreach_Dashboard/Universidade_Tecnol\%C3\%B3gica_Federal_do_Paran\%C3\%A1/Wikipedon_2020-02_(2020)}{Wikipedon}, uma parceria da UTFPR com a Fundação Wikimedia.

Enquanto nosso encontro não chega, aproveite para se familiarizar com o Sistema Brasileiro de Classificação de Solos. Afinal de contas, ele será nosso companheiro de trabalho por quatro meses.

Bom trabalho e até breve!

\textbf{NOTA 1:} Pode ocorrer sobrecarga da \href{https://pt.wikipedia.org/wiki/Rede_Nacional_de_Ensino_e_Pesquisa}{Rede Nacional de Ensino e Pesquisa} (RNP) durante nosso encontro. Caso isso ocorra, utilizaremos uma sala alternativa do Google Classroom (\href{https://meet.google.com/lookup/c4saebrx42}{Pressione aqui para acessar}).

\textbf{NOTA 2:} Não se preocupe caso não consiga participar do encontro síncrono. Nosso encontro será gravado e ficará disponível aqui no Moodle para você assistir quando tiver condições.

\textbf{NOTA 3:} O encontro será realizado em sala virtual do Moodle, utilizando o \emph{plugin} BibBlueButtonBN e a infraestrutura da \href{https://pt.wikipedia.org/wiki/Rede_Nacional_de_Ensino_e_Pesquisa}{Rede Nacional de Ensino e Pesquisa} (RNP). A sala virtual será aberta às 10 h e 55 min. Para acessar, basta pressionar no título dessa atividade logo acima.

\hypertarget{questionuxe1rio}{%
\section{Questionário}\label{questionuxe1rio}}

Gênese e morfologia do solo (10 min)

O questionário desta semana faz parte da primeira avaliação individual da disciplina AG63C Classificação de Solos. O objetivo é aferir seu conhecimento sobre gênese e morfologia do solo. São 10 questões a serem respondidas em até 10 minutos. Algumas dessas questões vão exigir menos de você do que outras. Assim, você tem, em média, um minuto para responder a cada questão. Caso seu desempenho não seja satisfatório, você dispõe de mais duas tentativas para melhorar sua nota. A cada tentativa, outras 10 novas questões, selecionadas aleatoriamente de um banco de questões, lhe serão apresentadas. A nota final será a média aritmética de todas as suas tentativas.

Antes de responder ao questionário, você pode refrescar a memória revendo suas anotações pessoais ou o material disponibilizado pelo professor da disciplina \href{http://portal.utfpr.edu.br/cursos/coordenacoes/graduacao/santa-helena/sh-agronomia/matriz-e-docentes}{AG62B Gênese e Morfologia do Solo}. Você também pode reler a seção \emph{1.1 Caracterização Morfológica e Descrição dos Solos} do \href{https://biblioteca.ibge.gov.br/visualizacao/livros/liv95017.pdf}{Manual Técnico de Pedologia} (IBGE, 2015, p.~39-151).

Bom trabalho e até breve!

\hypertarget{semana-02}{%
\chapter{Semana 02}\label{semana-02}}

Dados e informações usados na classificação de solos

\hypertarget{duxfavidas-da-semana}{%
\section{Dúvidas da semana}\label{duxfavidas-da-semana}}

Você tem dúvidas sobre as atividades da semana anterior? Quer saber mais sobre as atividades que terá que desenvolver na semana que está iniciando? Participe de nosso encontro semanal de resolução de dúvidas. Você também pode deixar uma mensagem no Fórum de Dúvidas e Sugestões.

\begin{itemize}
\tightlist
\item
  Informações gerais

  \begin{itemize}
  \tightlist
  \item
    Local: Moodle
  \item
    Horário: 11:00--11:50
  \item
    Duração prevista: 50 min
  \item
    Gravação: não
  \end{itemize}
\end{itemize}

\hypertarget{leitura}{%
\section{Leitura}\label{leitura}}

Manual Técnico de Pedologia (37 min)

A leitura obrigatória desta semana traz a seção \emph{1.2.2 Principais determinações e métodos de análises utilizados em levantamentos de solos no Brasil} do \href{https://biblioteca.ibge.gov.br/visualizacao/livros/liv95017.pdf}{Manual Técnico de Pedologia} (IBGE, 2015, p.~151-169). Ela tem como objetivo fazer com que você se familiarize com essas determinações e métodos de análise. Conhecê-las é de fundamental importância, pois a classificação de solos requer a caracterização completa do solo. E isso só pode ser feito enviando amostras para laboratórios de análise qualificados.

À primeira vista, o tema pode lhe parecer complexo, sobretudo porque você ainda não deve ter cursado nenhuma disciplina que tratasse especificamente dos métodos de análise química e física do solo. Contudo, como foi dito acima, o objetivo é que você apenas se familiarize com as principais determinações e métodos de análise. Sem isso, você terá bastante dificuldade para utilizar o Sistema Brasileiro de Classificação de Solos.

Boa leitura e até breve!

\hypertarget{questionuxe1rio-1}{%
\section{Questionário}\label{questionuxe1rio-1}}

Dados para classificação de solos (10 min)

Este questionário também faz parte da primeira avaliação individual da nossa disciplina. Aqui, o objetivo é aferir seu conhecimento sobre os dados morfológicos, químicos e físicos utilizados para a classificação de solos. Assim como no questionário anterior, são 10 questões a serem respondidas em até 10 minutos. Você continua dispondo de três tentativas para responder ao questionário, sendo sua nota dada pela média aritmética de todas as tentativas.

Você não precisa se preocupar com o fato de ainda não cursado disciplinas que tratassem especificamente de métodos de análise química e física do solo. A leitura atenta da seção \emph{1.2.2 Principais Determinações e Métodos de Análises Utilizados em Levantamentos de Solos no Brasil} do \href{https://biblioteca.ibge.gov.br/visualizacao/livros/liv95017.pdf}{Manual Técnico de Pedologia} (IBGE, 2015, p.~151-169) é suficiente para você ter sucesso na resolução do questionário.

Bom trabalho e até breve!

\hypertarget{fuxf3rum-i}{%
\section{Fórum I}\label{fuxf3rum-i}}

Fontes de dados e informações do solo (20 min)

Você já deve saber que o seu papel como engenheiro(a) será o de resolver problemas agronômicos de interesse da sociedade. A forma mais eficiente de fazer isso é aplicando os princípios do método científico. Independente do problema que você terá que resolver, um dos primeiros passos vai sempre consistir em compilar os dados e informações existentes sobre o problema à mão. Depois de analisar esses dados e informações, você terá condições de decidir se necessita coletar mais dados ou se já consegue formular uma resposta satisfatória o problema.

Neste primeiro fórum, que tem duração de três semana, o objetivo é responder a duas questões fundamentais. Primeiro, quais são as estratégias mais eficientes para encontrar fontes de dados e informações de recursos do solo e da terra de uma determinada região geográfica? Segundo, quais são as características que permitem identificar que uma dessas fontes é confiável?

Lembre-se de que você precisa interagir com seus colegas, pelo menos, uma vez por semana para completar essa atividade.

Bom trabalho e até breve!

\hypertarget{encontro-1}{%
\section{Encontro}\label{encontro-1}}

Organização de informações de recursos do solo e da terra (50 min)

O objetivo dessa disciplina é que você chegue ao final dela sendo capaz de organizar informações sobre os recursos do solo e da terra de regiões geográficas. Em outras palavas, que você seja capaz de encontrar fontes confiáveis de dados e informação, sumariar as informações encontradas, reportar as relações existentes entre os componentes do meio físico, e produzir conteúdo digital para comunicação. O projeto de extensão Wikipedon foi criado exatamente para te ajudar a desenvolver essa capacidade.

Em nosso segundo encontro, vamos conversar sobre como organizar informações de recursos do solo e da terra para publicação na Wikipédia. Vamos abordar tópicos como as páginas de teste e o editor de texto da Wikipédia. Veremos também como o WMF Outreach Dashboard nos auxiliará na organização do trabalho em grupo. Enquanto nosso encontro não chega, você pode ir se familiarizando com a \href{https://pt.wikiversity.org/wiki/Ajuda:Como_editar_em_um_projeto_Wiki}{edição de projetos wiki} e conhecer mais sobre o uso da \href{https://pt.wikiversity.org/wiki/Oficinas_Wikimedia_\%26_Educa\%C3\%A7\%C3\%A3o}{Wikipédia na educação}.

\hypertarget{wikipedon}{%
\section{Wikipedon}\label{wikipedon}}

Formação dos grupos de trabalho (10 min)
\url{https://outreachdashboard.wmflabs.org/courses/Universidade_Tecnol\%C3\%B3gica_Federal_do_Paran\%C3\%A1/Wikipedon_2020-02_(2020)}

Essa é a primeira etapa do \href{https://pt.wikiversity.org/wiki/Wikipedon}{trabalho em grupo} e algumas tarefas importante precisam ser cumpridas. A primeira delas consiste exatamente em constituir seu grupo de trabalho. Para isso, você precisa identificar dois colegas com os quais gostaria de trabalhar durante todo o semestre letivo. Em seguida, você e seus colegas de trabalho devem escolher um dos quinze \href{https://outreachdashboard.wmflabs.org/courses/Universidade_Tecnol\%C3\%B3gica_Federal_do_Paran\%C3\%A1/Wikipedon_2020-02_(2020)/articles/available}{municípios paranaenses} participantes do projeto Wikipedon. Esse é o município para o qual seu grupo terá que produzir conteúdo digital sobre os recursos do solo e da terra para publicação na Wikipédia.

Depois de formado o grupo de trabalho e selecionado o município de interesse, você precisa se inscrever no \href{https://outreachdashboard.wmflabs.org/courses/Universidade_Tecnológica_Federal_do_Paraná/Wikipedon_2020-02_(2020)?locale=pt-br\&enroll=nitossolo?locale=pt-br}{WMF Outreach Dashboard}. O WMF Outreach Dashboard é a plataforma que nós utilizaremos para gerir o trabalho em grupo (você precisa ter uma \href{https://pt.wikipedia.org/w/index.php?title=Especial:Criar_conta}{conta de usuário da Wikipédia} para se inscrever). Completada a inscrição, você só precisa registrar no WMF Outreach Dashboard o município com o qual vai trabalhar. Fique tranquilo pois nós preparamos um \href{https://pt.wikiversity.org/wiki/Wikipedon}{tutorial} para lhe auxiliar nessa tarefa.

Bom trabalho e até breve!

\hypertarget{semana-03}{%
\chapter{Semana 03}\label{semana-03}}

Fontes de dados e informações do solo

Nesta semana, você conhecerá algumas das mais importantes fontes de dados de recursos do solo do Brasil e do mundo. Você vai conhecer as características desejáveis de uma base pública de dados do solo, bem como maneiras alternativas de produção de dados do solo. Também lhe serão apresentadas as três leis e os oito princípios dos dados abertos, e como cada repositório de dados do solo atende a essas leis e princípios. Ao final da semana, espera-se que você seja capaz de selecionar fontes confiáveis de dados de recursos do solo de regiões geográficas do território brasileiro.

Para alcançar esse objetivo, você deve fazer a leitura de dois artigos publicados no \href{https://www.sbcs.org.br/?post_type=boletim}{Boletim Informativo} da Sociedade Brasileira de Ciência do Solo (\href{https://www.sbcs.org.br}{SBCS}). Esses artigos servirão de base para você responder ao questionário da semana, momento em que você poderá verificar seu aprendizado. Eles também proverão elementos para sua segunda participação no fórum sobre fontes de dados e informações do solo. Finalmente, você e seus colegas de trabalho encontrarão pistas importantes sobre onde buscar dados dos recursos do solo e da terra do município com o qual estão trabalhando no projeto \href{https://pt.wikiversity.org/wiki/Wikipedon}{Wikipedon}.

\hypertarget{duxfavidas-da-semana-1}{%
\section{Dúvidas da semana}\label{duxfavidas-da-semana-1}}

Você tem dúvidas sobre as atividades da semana anterior? Quer saber mais sobre as atividades que terá que desenvolver na semana que está iniciando? Participe de nosso encontro semanal de resolução de dúvidas. Você também pode deixar uma mensagem no Fórum de Dúvidas e Sugestões.

\begin{itemize}
\tightlist
\item
  Informações gerais

  \begin{itemize}
  \tightlist
  \item
    Local: Moodle
  \item
    Horário: 11:00--11:50
  \item
    Duração prevista: 50 min
  \item
    Gravação: não
  \end{itemize}
\end{itemize}

\hypertarget{leitura-1}{%
\section{Leitura}\label{leitura-1}}

Boletim Informativo da SBCS (7 min + 15 min)
* SAMUEL-ROSA, A.; VASQUES, G. M. Dados para aplicações pedométricas em larga escala no Brasil. Boletim Informativo da SBCS, v. 43, n.~3, p.~22--24, 2017. Disponível em: \url{https://www.sbcs.org.br/wp-content/uploads/2018/01/boletimsbcs32017ebook_03_01_2018_10_45_30_id_36404.pdf}
* SAMUEL-ROSA, A. Vamos abrir os dados da pesquisa sobre o solo? Boletim Informativo da SBCS, v. 45, n.~1, p.~27--31, 2019. Disponível em: \url{https://www.sbcs.org.br/wp-content/uploads/2019/06/Boletim-SBCS-Volume-45-N\%C3\%BAmero-1.pdf}

\hypertarget{questionuxe1rio-2}{%
\section{Questionário}\label{questionuxe1rio-2}}

Fontes de dados e informações do solo (10 min)

\hypertarget{fuxf3rum-i-1}{%
\section{Fórum I}\label{fuxf3rum-i-1}}

Fontes de dados e informações do solo (10 min)
Quais são as estratégias mais eficientes para encontrar fontes de dados e informações de recursos do solo e da terra de uma determinada região geográfica? Quais são as características que permitem identificar que uma dessas fontes é confiável? (\href{https://moodle.utfpr.edu.br/mod/forum/view.php?id=549655}{Acessar o fórum})

\hypertarget{wikipedon-1}{%
\section{Wikipedon}\label{wikipedon-1}}

Identificar fontes de dados informações (20 min)
Identificar fontes de dados e informações dos recursos do solo e da terra do município que você e seus colegas de trabalho escolheram, usando a respectiva página de rascunho de teste para sua organização. Mais informações na página do projeto \href{https://pt.wikiversity.org/wiki/Wikipedon}{Wikipedon}.

\hypertarget{recursos-adicionais}{%
\section{Recursos adicionais}\label{recursos-adicionais}}

Palestra sobre acesso aberto aos dados da pesquisa em ciência do solo (SBCS) (1,5 h)

\url{https://www.youtube.com/watch?v=xAW0JktjlOI\&t=1758s}

\hypertarget{semana-04}{%
\chapter{Semana 04}\label{semana-04}}

SiBCS: estrutura e funcionamento

A quarta semana de atividades ficará marcada como sendo aquela em que você conhecerá a estrutura e funcionamento do Sistema Brasileiro de Classificação de Solos (\href{https://www.embrapa.br/en/solos/sibcs}{SiBCS}). Para isso, você deve fazer a leitura de parte do Manual Técnico de Pedologia (IBGE, 2015, p.~185-203). Lá você vai conhecer um pouco da história do SiBCS e como ele chegou a ter a estrutura multi categórica, hierárquica e aberta que possui hoje. Você também vai conhecer os seis níveis categóricos do SiBCS e a nomenclatura utilizada em cada um deles, tendo a oportunidade de entender as razões por traz da formulação dos termos mais comumente utilizados. Ao final da semana, espera-se que você seja capaz de apontar as principais características sobre o funcionamento e estrutura do SiBCS---capacidade que será avaliada em nosso questionário semanal.

Durante esta semana você também deverá iniciar a redação do texto sobre os recursos do solo e da terra do município que você e seus colegas de trabalho escolheram. Para isso, vocês devem utilizar a página de rascunho de teste daquele município. Se você tiver dúvidas sobre como proceder, comece procurando por respostas na página do projeto \href{https://pt.wikiversity.org/wiki/Wikipedon}{Wikipedon}. Você também pode participar de nosso encontro para resolução das dúvidas da semana (veja abaixo) ou deixar uma mensagem no \href{https://moodle.utfpr.edu.br/mod/forum/view.php?id=573329}{fórum de dúvidas e sugestões}.

\hypertarget{duxfavidas-da-semana-2}{%
\section{Dúvidas da semana}\label{duxfavidas-da-semana-2}}

Você tem dúvidas sobre as atividades da semana anterior? Quer saber mais sobre as atividades que terá que desenvolver na semana que está iniciando? Participe de nosso encontro semanal de resolução de dúvidas. Você também pode deixar uma mensagem no Fórum de Dúvidas e Sugestões.

\begin{itemize}
\tightlist
\item
  Informações gerais

  \begin{itemize}
  \tightlist
  \item
    Local: Moodle
  \item
    Horário: 11:00--11:50
  \item
    Duração prevista: 50 min
  \item
    Gravação: não
  \end{itemize}
\end{itemize}

\hypertarget{leitura-2}{%
\section{Leitura}\label{leitura-2}}

Manual Técnico de Pedologia (32 min)

\begin{itemize}
\tightlist
\item
  IBGE. \textbf{Capítulo 2.1 Taxonomia de solos}. In: IBGE. Manual Técnico de Pedologia. 3. ed.~Rio de Janeiro, RJ: Instituto Brasileiro de Geografia e Estatística, 2015. p.~185-203. URL: \url{https://biblioteca.ibge.gov.br/visualizacao/livros/liv95017.pdf}
\end{itemize}

\hypertarget{questionuxe1rio-3}{%
\section{Questionário}\label{questionuxe1rio-3}}

SiBCS: estrutura e funcionamento (10 min)

\hypertarget{fuxf3rum-i-2}{%
\section{Fórum I}\label{fuxf3rum-i-2}}

Fontes de dados e informações do solo (10 min)

Quais são as estratégias mais eficientes para encontrar fontes de dados e informações de recursos do solo e da terra de uma determinada região geográfica? Quais são as características que permitem identificar que uma dessas fontes é confiável? (\href{https://moodle.utfpr.edu.br/mod/forum/view.php?id=549655}{Acessar o fórum})

\hypertarget{wikipedon-2}{%
\section{Wikipedon}\label{wikipedon-2}}

Compilar dados e informações encontrados (40 min)

Compilar, na página de rascunho de teste do município escolhido, os dados e informações sobre os recursos do solo e da terra que você e seus colegas de trabalho encontraram. Mais informações na página do projeto \href{https://pt.wikiversity.org/wiki/Wikipedon}{Wikipedon}.

\hypertarget{semana-05}{%
\chapter{Semana 05}\label{semana-05}}

SiBCS: atributos diagnósticos

A última semana de atividades foi aquela em que você conheceu a estrutura e funcionamento do Sistema Brasileiro de Classificação de Solos (\href{https://www.embrapa.br/en/solos/sibcs}{SiBCS}). Nesta semana, você conhecerá um elemento importante do SiBCS, os chamados \emph{atributos diagnósticos}, e aprenderá como são utilizados na classificação de solos. Para isso, você deve fazer a leitura de parte do Manual Técnico de Pedologia (IBGE, 2015, p.~204-228). Você verá que há bastante \href{https://pt.wikipedia.org/wiki/Racioc\%C3\%ADnio_l\%C3\%B3gico-matem\%C3\%A1tico}{raciocínio lógico-matemático} envolvido na definição dos atributos diagnósticos. Ao final da semana, espera-se que você seja capaz de \textbf{aplicar} o SiBCS para identificação de atributos diagnósticos em perfis do solo. Essa capacidade será avaliada em nosso questionário semanal, que utilizará perfis de solo simulados.

Durante esta semana, você e seus colegas darão continuidade ao trabalho em grupo no projeto \href{https://pt.wikiversity.org/wiki/Wikipedon}{Wikipedon}. A tarefa desta semana consiste em contactar profissionais das ciências agrárias ou geociências com conhecimento prático do município escolhido por seu grupo. O principal objetivo do contato é convidar alguns especialistas locais para acompanhar a compilação de dados e informações sobre os recursos do solo e da terra daquele município. Se você tiver dúvidas sobre como proceder, comece procurando por respostas na página do projeto \href{https://pt.wikiversity.org/wiki/Wikipedon}{Wikipedon}. Você também pode participar de nosso encontro para resolução das dúvidas da semana (veja abaixo) ou deixar uma mensagem no \href{https://moodle.utfpr.edu.br/mod/forum/view.php?id=573329}{fórum de dúvidas e sugestões}.

\hypertarget{duxfavidas-da-semana-3}{%
\section{Dúvidas da semana}\label{duxfavidas-da-semana-3}}

Você tem dúvidas sobre as atividades da semana anterior? Quer saber mais sobre as atividades que terá que desenvolver na semana que está iniciando? Participe de nosso encontro semanal de resolução de dúvidas. Você também pode deixar uma mensagem no Fórum de Dúvidas e Sugestões.

\begin{itemize}
\tightlist
\item
  Informações gerais

  \begin{itemize}
  \tightlist
  \item
    Local: Moodle
  \item
    Horário: 11:00--11:50
  \item
    Duração prevista: 50 min
  \item
    Gravação: não
  \end{itemize}
\end{itemize}

\hypertarget{leitura-3}{%
\section{Leitura}\label{leitura-3}}

Manual Técnico de Pedologia (43 min)
* IBGE. \textbf{Seção 2.2.1.1 Atributos diagnósticos}. In: IBGE. Manual Técnico de Pedologia. 3. ed.~Rio de Janeiro, RJ: Instituto Brasileiro de Geografia e Estatística, 2015. p.~204-227. URL: \url{https://biblioteca.ibge.gov.br/visualizacao/livros/liv95017.pdf}
* IBGE. \textbf{Seção 2.2.1.2 Outros atributos}. In: IBGE. Manual Técnico de Pedologia. 3. ed.~Rio de Janeiro, RJ: Instituto Brasileiro de Geografia e Estatística, 2015. p.~227-228. URL: \url{https://biblioteca.ibge.gov.br/visualizacao/livros/liv95017.pdf}

\hypertarget{questionuxe1rio-4}{%
\section{Questionário}\label{questionuxe1rio-4}}

SiBCS: atributos diagnósticos (10 min)

\hypertarget{wikipedon-3}{%
\section{Wikipedon}\label{wikipedon-3}}

Contactar especialistas locais (15 min)

Tarefa

\textbf{Contactar} profissionais das ciências agrárias ou geociências com experiência no município escolhido por seu grupo, convidando-os para avaliar a acurácia (qualidade) e suficiência (quantidade) dos dados e informações sobre os recursos do solo e da terra compilados para aquele município.

Objetivo

\textbf{Validar} a compilação de dados e informações sobre os recursos do solo e da terra, avaliando se o material compilado é acurado e suficiente para produzir conteúdo digital para o município escolhido por seu grupo.

Procedimentos

O contato deve ser realizado via correio eletrônico ou mídias sociais. Os conteúdos de texto e/ou áudio e/ou vídeo utilizados no contato devem ser salvos em \href{https://pt.wikipedia.org/wiki/Compactador_de_arquivos}{arquivo compactado} (.7z .bdoc .cdoc .ddoc .gtar .gz .gzip .hqx .rar .sit .tar .tgz .zip), junto de arquivo de texto contendo os dados de identificação, endereço e contato dos especialistas. A comprovação da conclusão da atividade será mediante o envio do arquivo compactado. Mais informações na página do projeto \href{https://pt.wikiversity.org/wiki/Wikipedon}{Wikipedon}.

\hypertarget{recursos-adicionais-1}{%
\section{Recursos adicionais}\label{recursos-adicionais-1}}

Aula sobre atributos diagnósticos (UFPR) (24 min)

\url{https://youtu.be/aLhgbAB2FDY}

Resumo sobre atributos diagnósticos (UFSM) (6 min)

\url{https://youtu.be/9qLrWBGnvAY}

\hypertarget{semana-06}{%
\chapter{Semana 06}\label{semana-06}}

SiBCS: horizontes diagnósticos

Nesta semana, você começará a conhecer mais um elemento importante do funcionamento do Sistema Brasileiro de Classificação de Solos (\href{https://www.embrapa.br/en/solos/sibcs}{SiBCS}): os \emph{horizontes diagnósticos}. Para isso, você deve fazer a leitura de parte do Manual Técnico de Pedologia (IBGE, 2015, p.~229-234) que trata dos \emph{horizontes diagnósticos superficiais}. Você verá que, assim como para os atributos diagnósticos estudados na última semana de atividades, há bastante \href{https://pt.wikipedia.org/wiki/Racioc\%C3\%ADnio_l\%C3\%B3gico-matem\%C3\%A1tico}{raciocínio lógico-matemático} envolvido na definição dos horizontes diagnósticos. Você também terá que fazer a leitura de um trecho do livro \href{https://www.sbcs.org.br/loja/index.php?route=product/product\&product_id=57}{Pedologia: Fundamentos} (RESENDE et al., 2012), editado pela Sociedade Brasileira de Ciência do Solo.

Na semana passada, você e seus colegas contactaram profissionais das ciências agrárias ou geociências com conhecimento prático do município escolhido por seu grupo no projeto \href{https://pt.wikiversity.org/wiki/Wikipedon}{Wikipedon}. Em alguns casos, o retorno de um contato feito pode demorar. Em outros, pode ser preciso insistir no contato ou procurar por outros especialistas. Se o contato foi bem sucedido, então será preciso dar continuidade à conversa e explicar aos especialistas sobre o escopo do trabalho e como esperamos que auxiliem. Por isso, durante essa semana, você e seus colegas darão continuidade à atividade iniciada na semana passada. Lembre-se de que o objetivo é encontrar especialistas que possam acompanhar e contribuir para a compilação de dados e informações daquele município. Se você tiver dúvidas sobre como proceder, comece procurando por respostas na página do projeto \href{https://pt.wikiversity.org/wiki/Wikipedon}{Wikipedon}. Você também pode participar de nosso encontro para resolução das dúvidas da semana (veja abaixo) ou deixar uma mensagem no \href{https://moodle.utfpr.edu.br/mod/forum/view.php?id=573329}{fórum de dúvidas}.

\hypertarget{duxfavidas-da-semana-4}{%
\section{Dúvidas da semana}\label{duxfavidas-da-semana-4}}

Você tem dúvidas sobre as atividades da semana anterior? Quer saber mais sobre as atividades que terá que desenvolver na semana que está iniciando? Participe de nosso encontro semanal de resolução de dúvidas. Você também pode deixar uma mensagem no Fórum de Dúvidas e Sugestões.

\begin{itemize}
\tightlist
\item
  Informações gerais

  \begin{itemize}
  \tightlist
  \item
    Local: Moodle
  \item
    Horário: 11:00--11:50
  \item
    Duração prevista: 50 min
  \item
    Gravação: não
  \end{itemize}
\end{itemize}

\hypertarget{leitura-4}{%
\section{Leitura}\label{leitura-4}}

Manual Técnico de Pedologia (8 min)

\begin{itemize}
\tightlist
\item
  IBGE. \textbf{Seção 2.2.1.3 Horizontes diagnósticos superficiais}. In: IBGE. Manual Técnico de Pedologia. 3. ed.~Rio de Janeiro, RJ: Instituto Brasileiro de Geografia e Estatística, 2015. p.~229-234. URL: \url{https://biblioteca.ibge.gov.br/visualizacao/livros/liv95017.pdf}
\end{itemize}

\hypertarget{leitura-5}{%
\section{Leitura}\label{leitura-5}}

Princípios da Classificação dos Solos (15 min)

\begin{itemize}
\tightlist
\item
  RESENDE, M. et al.~\textbf{Princípios da classificação dos solos}. In: KER, J. C. et al.~(Eds.) Pedologia: Fundamentos. Viçosa: Sociedade Brasileira de Ciência do Solo, 2012. p.~21--46. URL: \url{https://www.dropbox.com/s/1obglu7socskkla/ResendeEtAl2012_Princ\%C3\%ADpios_da_classifica\%C3\%A7\%C3\%A3o_dos_solos.pdf?dl=0}
\end{itemize}

\hypertarget{wikipedon-4}{%
\section{Wikipedon}\label{wikipedon-4}}

(Re)Contactar especialistas locais (15 min)

\hypertarget{encontro-2}{%
\section{Encontro}\label{encontro-2}}

Wikipedon: balanço de atividades e próximos passos (50 min)

\hypertarget{recursos-adicionais-2}{%
\section{Recursos adicionais}\label{recursos-adicionais-2}}

Aula sobre horizontes diagnósticos (UFPR) (17 min)

\url{https://youtu.be/CQ3ruCSUaWE}

Resumo sobre horizontes diagnósticos (InteraSolo) (3 min)

\url{https://youtu.be/rrGkPzHeezM}

\hypertarget{semana-07}{%
\chapter{Semana 07}\label{semana-07}}

SiBCS: horizontes diagnósticos superficiais

Nesta semana, você conhecerá ainda mais sobre os \emph{horizontes diagnósticos superficiais}, um elemento muito importante do funcionamento do Sistema Brasileiro de Classificação de Solos (\href{https://www.embrapa.br/en/solos/sibcs}{SiBCS}). Ao final da semana, espera-se que você seja capaz de \textbf{aplicar} o SiBCS para identificação de horizontes diagnósticos em perfis do solo.

Na semana passada, você e seus colegas de trabalho deram continuidade ao contato com pessoas com conhecimento prático do município escolhido. Se o contato foi bem sucedido, vocês devem dar continuidade à conversa e compilar os novos dados e informações disponibilizados. Caso contrário, vocês devem procurar novamente por fontes confiáveis de dados e informações dos recursos do solo e da terra. O \textbf{objetivo} é chegar ao final desta semana de atividades tendo em mãos a maior parte dos dados e informações necessários para concluir o trabalho.

Se você tiver dúvidas sobre como proceder, comece procurando por respostas na página do projeto \href{https://pt.wikiversity.org/wiki/Wikipedon}{Wikipedon}. Você também pode participar de nosso encontro para resolução das dúvidas da semana (veja abaixo) ou deixar uma mensagem no \href{https://moodle.utfpr.edu.br/mod/forum/view.php?id=573329}{fórum de dúvidas}.

\hypertarget{duxfavidas-da-semana-5}{%
\section{Dúvidas da semana}\label{duxfavidas-da-semana-5}}

Você tem dúvidas sobre as atividades da semana anterior? Quer saber mais sobre as atividades que terá que desenvolver na semana que está iniciando? Participe de nosso encontro semanal de resolução de dúvidas. Você também pode deixar uma mensagem no Fórum de Dúvidas e Sugestões.

\begin{itemize}
\tightlist
\item
  Informações gerais

  \begin{itemize}
  \tightlist
  \item
    Local: Moodle
  \item
    Horário: 11:00--11:50
  \item
    Duração prevista: 50 min
  \item
    Gravação: não
  \end{itemize}
\end{itemize}

\hypertarget{questionuxe1rio-5}{%
\section{Questionário}\label{questionuxe1rio-5}}

SiBCS: atributos diagnósticos (25 min)

IMPORTANTE: Ao final do questionário, você terá que comprovar que realizou os cálculos para avaliação dos atributos diagnósticos do perfil de solo simulado. Você poderá carregar uma imagem, uma planilha ou qualquer outro arquivo que julgar pertinente.

\hypertarget{wikipedon-5}{%
\section{Wikipedon}\label{wikipedon-5}}

Compilar (novos) dados e informações encontrados (20 min)

\hypertarget{fuxf3rum-ii}{%
\section{Fórum II}\label{fuxf3rum-ii}}

Fontes de dados e informações do solo revisitadas (30 min)

O \href{https://moodle.utfpr.edu.br/mod/forum/view.php?id=549655}{primeiro fórum} teve como objetivo responder a uma questão fundamental:

\begin{quote}
Quais são as estratégias mais eficientes para encontrar fontes confiáveis de dados e informações de recursos do solo e da terra de determinada região geográfica?
\end{quote}

Nas semanas que seguiram ao término do primeiro fórum, você e seus colegas trabalharam para compilar dados e informações de recursos do solo e da terra do município escolhido. Vocês também contactaram munícipes e especialistas das geociências e ciências agrárias para obter mais dados e informações, bem como suporte na organização dos dados e informações compilados. Será que essa experiência prática pode lhe ajudar a rever, sob outra perspectiva, alguns pontos discutidos no primeiro fórum?

Neste segundo fórum, você tem as seguintes tarefas:

\begin{enumerate}
\def\labelenumi{\arabic{enumi}.}
\tightlist
\item
  Reler as postagens suas e de seus colegas no primeiro fórum;
\item
  Analisar, criticamente, suas postagens e de seus colegas, considerando as experiências práticas das últimas semanas;
\item
  Publicar uma postagem com seu entendimento atual sobre as estratégias mais eficientes para encontrar fontes confiáveis de dados e informações de recursos do solo e da terra de determinada região geográfica.
\end{enumerate}

\hypertarget{encontro-3}{%
\section{Encontro}\label{encontro-3}}

SiBCS: horizontes diagnósticos I (50 min)

\hypertarget{recursos-adicionais-3}{%
\section{Recursos adicionais}\label{recursos-adicionais-3}}

Aula sobre horizontes diagnósticos superficiais (UFSM) (9 min)

\url{https://youtu.be/LYrOGzAt8dY}

\hypertarget{semana-08}{%
\chapter{Semana 08}\label{semana-08}}

SiBCS: horizontes diagnósticos subsuperficiais

Nesta semana, você conhecerá mais sobre os \emph{horizontes diagnósticos subsuperficiais}, um elemento muito importante do funcionamento do Sistema Brasileiro de Classificação de Solos (\href{https://www.embrapa.br/en/solos/sibcs}{SiBCS}). Ao final da semana, espera-se que você seja capaz de \textbf{aplicar} o SiBCS para identificação de horizontes diagnósticos em perfis do solo.

Se você tiver dúvidas sobre como proceder, comece procurando por respostas na página do projeto \href{https://pt.wikiversity.org/wiki/Wikipedon}{Wikipedon}. Você também pode participar de nosso encontro para resolução das dúvidas da semana (veja abaixo) ou deixar uma mensagem no \href{https://moodle.utfpr.edu.br/mod/forum/view.php?id=573329}{fórum de dúvidas}.

\hypertarget{duxfavidas-da-semana-6}{%
\section{Dúvidas da semana}\label{duxfavidas-da-semana-6}}

Você tem dúvidas sobre as atividades da semana anterior? Quer saber mais sobre as atividades que terá que desenvolver na semana que está iniciando? Participe de nosso encontro semanal de resolução de dúvidas. Você também pode deixar uma mensagem no Fórum de Dúvidas e Sugestões.

\begin{itemize}
\tightlist
\item
  Informações gerais

  \begin{itemize}
  \tightlist
  \item
    Local: Moodle
  \item
    Horário: 11:00--11:50
  \item
    Duração prevista: 50 min
  \item
    Gravação: não
  \end{itemize}
\end{itemize}

\hypertarget{leitura-6}{%
\section{Leitura}\label{leitura-6}}

Manual Técnico de Pedologia (43 min)

\begin{itemize}
\tightlist
\item
  IBGE. \textbf{Seção 2.2.1.4 Horizontes diagnósticos subsuperficiais}. In: IBGE. Manual Técnico de Pedologia. 3. ed.~Rio de Janeiro, RJ: Instituto Brasileiro de Geografia e Estatística, 2015. p.~234-251. URL: \url{https://biblioteca.ibge.gov.br/visualizacao/livros/liv95017.pdf}
\item
  IBGE. \textbf{Seção 2.2.1.5 Outros horizontes diagnósticos subsuperficiais}. In: IBGE. Manual Técnico de Pedologia. 3. ed.~Rio de Janeiro, RJ: Instituto Brasileiro de Geografia e Estatística, 2015. p.~252-257. URL: \url{https://biblioteca.ibge.gov.br/visualizacao/livros/liv95017.pdf}
\end{itemize}

\hypertarget{questionuxe1rio-6}{%
\section{Questionário}\label{questionuxe1rio-6}}

SiBCS: horizontes diagnósticos (10 min)

\hypertarget{recursos-adicionais-4}{%
\section{Recursos adicionais}\label{recursos-adicionais-4}}

Aula sobre horizontes diagnósticos subsuperficiais (UFSM) (15 min)

\url{https://youtu.be/3LeyzQtKs-4}

Exercício sobre horizontes diagnósticos subsuperficiais (UFSM) (6 min)

\url{https://youtu.be/tRG2ybqIFQw}

\hypertarget{semana-09}{%
\chapter{Semana 09}\label{semana-09}}

SiBCS: características e distribuição espacial das 13 ordens

Você já sabe que o Sistema Brasileiro de Classificação de Solos (\href{https://www.embrapa.br/en/solos/sibcs}{SiBCS}) possui 13 ordens. Nesta semana, você conhecerá mais sobre as características e distribuição espacial de cada uma dessa ordens no território brasileiro. Isso será de grande ajuda na sumarização dos dados e informações sobre os recursos do solo do município com o qual você está trabalhando no projeto \href{https://pt.wikiversity.org/wiki/Wikipedon}{Wikipedon}.

\hypertarget{duxfavidas-da-semana-7}{%
\section{Dúvidas da semana}\label{duxfavidas-da-semana-7}}

Você tem dúvidas sobre as atividades da semana anterior? Quer saber mais sobre as atividades que terá que desenvolver na semana que está iniciando? Participe de nosso encontro semanal de resolução de dúvidas. Você também pode deixar uma mensagem no Fórum de Dúvidas e Sugestões.

\begin{itemize}
\tightlist
\item
  Informações gerais

  \begin{itemize}
  \tightlist
  \item
    Local: Moodle
  \item
    Horário: 11:00--11:50
  \item
    Duração prevista: 50 min
  \item
    Gravação: não
  \end{itemize}
\end{itemize}

\hypertarget{leitura-7}{%
\section{Leitura}\label{leitura-7}}

Manual Técnico de Pedologia (22 min)

\begin{itemize}
\tightlist
\item
  IBGE. \textbf{Seção 2.3 Principais solos do Brasil}. In: IBGE. Manual Técnico de Pedologia. 3. ed.~Rio de Janeiro, RJ: Instituto Brasileiro de Geografia e Estatística, 2015. p.~285-319. URL: \url{https://biblioteca.ibge.gov.br/visualizacao/livros/liv95017.pdf}
\end{itemize}

\hypertarget{leitura-8}{%
\section{Leitura}\label{leitura-8}}

Sistema Brasileiro de Classificação de Solos (27 min)

\begin{itemize}
\tightlist
\item
  Santos et al.~\textbf{Conceito e definição das classes do 1º nível categórico (ordens)}. In: Santos et al.~Sistema Brasileiro de Classificação de Solos. 5. ed.~Brasília, DF: Empresa Brasileira de Pesquisa Agropecuária, 2018. p.~86-106. URL: \url{https://ainfo.cnptia.embrapa.br/digital/bitstream/item/199517/1/SiBCS-2018-ISBN-9788570358004.pdf}
\end{itemize}

\hypertarget{questionuxe1rio-7}{%
\section{Questionário}\label{questionuxe1rio-7}}

SiBCS: horizontes diagnósticos (30 min)

Para responder ao questionário desta semana, você deve seguir os passos abaixo:

\begin{enumerate}
\def\labelenumi{\arabic{enumi}.}
\tightlist
\item
  Acesse o Repositório Brasileiro Livre para Dados Abertos do Solo (\href{https://www.pedometria.org/projeto/febr/}{FEBR})
\item
  No portal de busca e descarregamento de conjuntos de dados, acesse a aba de visualização espacial (mapa)
\item
  Navegue até o estado do Paraná e identifique os perfis do solo mais próximos do município com o qual você está trabalhando no Wikipedon
\item
  Selecione um desses perfis do solo e acesse o conjunto de dados na nuvem

  \begin{itemize}
  \tightlist
  \item
    Você e seus colegas de grupo devem selecionar perfis de solo distintos
  \end{itemize}
\item
  Identifique o horizonte diagnóstico superficial usando o fluxograma disponível no website do \href{https://www.pedometria.org/cursos/classificacao-de-solos/horizonte-diagnostico-superficial/}{Laboratório de Pedometria}
\end{enumerate}

Você deve tomar nota do caminho percorrido ao longo do fluxograma até identificar o horizonte diagnóstico superficial. Registre a pergunta de cada nó visitado e a resposta dada (TRUE ou FALSE). Sua resposta ao questionário deve conter o seguinte:

\begin{verbatim}
Título do conjunto de dados:
Código do conjunto de dados:
Código de identificação do perfil de solo:
Horizonte diagnóstico superficial identificado:
Caminho percorrido ao longo do fluxograma:

1. Há uma seção, superficial ou subsuperficial, em que Corg >= 80 g kg-1? TRUE
2. O Corg provém de acumulações naturais de resíduos vegetais? TRUE
3. A seção orgânica está assentada sobre rocha? FALSE
4. ...
\end{verbatim}

\hypertarget{fuxf3rum-iii}{%
\section{Fórum III}\label{fuxf3rum-iii}}

Buscando dados e informações de municípios vizinhos (20 min)

Dados e informações detalhados e atualizados sobre os recursos do solo e da terra são necessários para tomar decisões importantes que afetam o ambiente e a economia. Contudo, muitos municípios não dispõe desses dados e informações. Isso é o que ocorre, por exemplo, com o município de \href{https://pt.wikipedia.org/wiki/Corb\%C3\%A9lia}{Corbélia}, localizado na região metropolitana de Cascavel, estado do Paraná. Situação semelhante ocorre no caso da meteorologia. A maioria dos municípios não faz parte da rede de monitoramento meteorológico oficial. Nesses casos, a solução mais comumente usada consiste em buscar dados e informações em municípios vizinhos, que possuam condições ambientais similares. A hipótese que fundamenta essa prática postula que a proximidade geográfica e a similaridade de condições ambientais produzem condições meteorológicas similares. Seria possível adotar a mesma hipótese no caso dos recursos do solo? Discorra sobre os prós e contras dessa prática, tendo em vista os objetivos do projeto Wikipedon.

\hypertarget{recursos-adicionais-5}{%
\section{Recursos adicionais}\label{recursos-adicionais-5}}

Vídeos sobre características e distribuição espacial das 13 ordens do SiBCS

\url{https://www.youtube.com/playlist?list=PLJl3ZQuMf4knjfjN65KGIfwFDMRaDplMF}

\hypertarget{semana-10}{%
\chapter{Semana 10}\label{semana-10}}

SiBCS: Identificação das classes de solos

Você já conhece bastante sobre a estrutura do \href{https://www.embrapa.br/en/solos/sibcs}{SiBCS}. Agora está na hora de conhecer melhor como as classes de solo são identificadas. Como você já deve imaginar, isso depende da identificação prévia dos horizontes diagnósticos superficiais e subsuperficiais que estudamos nas semanas anteriores. Especificamente, você vai conhecer a chave de identificação das ordens, uma estrutura lógica similar àquelas com que você trabalhou em disciplinas das ciências biológicas. Além disso, vamos entrar numa nova fase do \href{https://pt.wikiversity.org/wiki/Wikipedon}{Wikipedon}. Nela, você deixará de lado (temporariamente) o município em que está trabalhando para contribuir no desenvolvimento do artigo de outro grupo. Isso permitirá que você se familiarize com os recursos do solo e da terra de outra região. Também permitirá que você desenvolva sua habilidade de extrapolação, aplicando experiências prévias em um novo cenário geoespacial. O aprofundamento do estudo do SiBCS e do paradigma da relação solo-paisagem serão de grande ajuda nesse processo.

\hypertarget{duxfavidas-da-semana-8}{%
\section{Dúvidas da semana}\label{duxfavidas-da-semana-8}}

Você tem dúvidas sobre as atividades da semana anterior? Quer saber mais sobre as atividades que terá que desenvolver na semana que está iniciando? Participe de nosso encontro semanal de resolução de dúvidas. Você também pode deixar uma mensagem no Fórum de Dúvidas e Sugestões.

\begin{itemize}
\tightlist
\item
  Informações gerais

  \begin{itemize}
  \tightlist
  \item
    Local: Moodle
  \item
    Horário: 11:00--11:50
  \item
    Duração prevista: 50 min
  \item
    Gravação: não
  \end{itemize}
\end{itemize}

\hypertarget{leitura-9}{%
\section{Leitura}\label{leitura-9}}

Sistema Brasileiro de Classificação de Solos (12 min)

\begin{itemize}
\tightlist
\item
  Santos, H. G. et al.~Capítulo 4 Classificação dos solos até o 4º nível categórico. In: Santos, H. G. et al.~\emph{Sistema Brasileiro de Classificação de Solos}. 5. ed.~Brasília, DF: Empresa Brasileira de Pesquisa Agropecuária, 2018. p.~129-138. URL: \url{https://ainfo.cnptia.embrapa.br/digital/bitstream/item/199517/1/SiBCS-2018-ISBN-9788570358004.pdf}
\end{itemize}

\hypertarget{leitura-10}{%
\section{Leitura}\label{leitura-10}}

Relação solo-paisagem (35 min)

\begin{itemize}
\tightlist
\item
  Campos, M. C. C. Relações solo-paisagem: conceitos, evolução e aplicações. \emph{Ambiência}, v. 8, n.~3, p.~963-982, 2012. URL: \url{https://revistas.unicentro.br/index.php/ambiencia/article/view/1290}
\end{itemize}

\hypertarget{fuxf3rum-iv}{%
\section{Fórum IV}\label{fuxf3rum-iv}}

Usando dados e informações de municípios vizinhos (20 min)

Dados e informações detalhados e atualizados sobre os recursos do solo e da terra são necessários para tomar decisões importantes que afetam o ambiente e a economia. Contudo, muitos municípios não dispõe desses dados e informações. O modo mais apropriado de atender a essa demanda consiste em realizar \href{https://www.cpt.com.br/cursos-administracaorural/artigos/levantamento-de-recursos-naturais-como-e-por-que-fazer}{levantamentos de campo} e construir uma \href{https://pt.wikipedia.org/wiki/Banco_de_dados_espaciais}{base de dados espaciais}. Isso geralmente requer grande volume de recursos---financeiros e humanos---e tempo. Assim, quando a demanda precisa ser atendida rapidamente---como é o caso do projeto Wikipedon---, costuma-se lançar mão de uma estratégia alternativa. Ela consiste em buscar dados e informações em áreas vizinhas que possuam condições ambientais similares. A \href{https://pt.wikipedia.org/wiki/Hip\%C3\%B3tese}{hipótese} que fundamenta essa prática é baseada na relação solo-paisagem. Segundo a \textbf{relação solo-paisagem}, podemos \href{https://pt.wikipedia.org/wiki/Previs\%C3\%A3o_do_tempo}{prever} as características do solo num local de interesse em \href{https://pt.wikipedia.org/wiki/Fun\%C3\%A7\%C3\%A3o_(matem\%C3\%A1tica)}{função} das condições ambientais em que se encontra. Para isso, é necessário, primeiro, conhecer as características do solo em locais próximos que possuam condições ambientais similares. Quanto maior proximidade e similaridade das condições ambientais dos dois locais, maior será a credibilidade das previsões.

Tendo em vista os objetivos do Wikipedon e o prazo de entrega da atividade, discorra sobre como a relação solo-paisagem pode ser explorada para melhorar o artigo do município sob sua revisão nesta nova fase do Wikipedon.

\hypertarget{wikipedon-6}{%
\section{Wikipedon}\label{wikipedon-6}}

Cooperar com pares na produção de conteúdo digital (30 min)

A \href{https://pt.wikipedia.org/wiki/Produ\%C3\%A7\%C3\%A3o_colaborativa}{produção colaborativa} do conhecimento sobre os recursos do solo e da terra é uma das maneiras mais eficientes de garantir sua qualidade. Nessa nova fase do Wikipedon, você deixará de lado (temporariamente) o município em que está trabalhando para se dedicar ao município de outro grupo. Você tem a tarefa de analisar o conteúdo compilado para aquele município, realizando as alterações pertinentes para melhorar a qualidade do artigo. Dois aspectos devem ser analisados. Primeiro, a qualidade dos dados e informações compilados---caso você conheça outras fontes, deve incluir as mesmas no artigo. Segundo, a qualidade da apresentação do texto---isso inclui o respeito às regras de formatação e inclusão de \href{https://pt.wikipedia.org/wiki/Wikip\%C3\%A9dia:Livro_de_estilo/Cite_as_fontes}{citações} e referências bibliográficas de modo apropriado, respeitando os direitos dos detentores de direito autoral. Ao final da semana, você deve apresentar um breve relato das contribuições feitas à melhoria do artigo.

\hypertarget{recursos-adicionais-6}{%
\section{Recursos adicionais}\label{recursos-adicionais-6}}

Vídeos sobre relação solo-paisagem

\url{https://www.youtube.com/playlist?list=PLJl3ZQuMf4kmAP_nfu5KCklYS0n0iM0MR}

\hypertarget{semana-11}{%
\chapter{Semana 11}\label{semana-11}}

Relação solo-paisagem

\hypertarget{duxfavidas-da-semana-9}{%
\section{Dúvidas da semana}\label{duxfavidas-da-semana-9}}

Você tem dúvidas sobre as atividades da semana anterior? Quer saber mais sobre as atividades que terá que desenvolver na semana que está iniciando? Participe de nosso encontro semanal de resolução de dúvidas. Você também pode deixar uma mensagem no Fórum de Dúvidas e Sugestões.

\begin{itemize}
\tightlist
\item
  Informações gerais

  \begin{itemize}
  \tightlist
  \item
    Local: Moodle
  \item
    Horário: 11:00--11:50
  \item
    Duração prevista: 50 min
  \item
    Gravação: não
  \end{itemize}
\end{itemize}

\hypertarget{questionuxe1rio-ix}{%
\section{Questionário IX}\label{questionuxe1rio-ix}}

SiBCS: horizontes diagnósticos (30 min)

Para responder ao questionário desta semana, você deve seguir os passos abaixo:

\begin{enumerate}
\def\labelenumi{\arabic{enumi}.}
\tightlist
\item
  Acesse o Repositório Brasileiro Livre para Dados Abertos do Solo (\href{https://www.pedometria.org/febr/buscar/}{FEBR})
\item
  Pesquise pelos conjuntos de dados do Projeto RADAMBRASIL, selecionando um dentre aqueles que não cobrem o estado do Paraná
\item
  Descarregue o arquivo XLSX do conjunto de dados
\item
  Selecione um perfil de solo sem comunicar aos seus colegas a sua escolha
\item
  Identifique o horizonte diagnóstico superficial usando o fluxograma disponível no website do \href{https://www.pedometria.org/cursos/classificacao-de-solos/horizonte-diagnostico-superficial/}{Laboratório de Pedometria}
\end{enumerate}

Você deve tomar nota do caminho percorrido ao longo do fluxograma até identificar o horizonte diagnóstico superficial. Registre a pergunta de cada nó visitado e a resposta dada (TRUE ou FALSE). Sua resposta ao questionário deve conter o seguinte:

\begin{verbatim}
ID do conjunto de dados:
Título do conjunto de dados:
ID do perfil de solo:
Horizonte diagnóstico superficial identificado:
Caminho percorrido ao longo do fluxograma:

1. Há uma seção, superficial ou subsuperficial, em que Corg >= 80 g kg-1? TRUE
2. O Corg provém de acumulações naturais de resíduos vegetais? TRUE
3. A seção orgânica está assentada sobre rocha? FALSE
4. ...
\end{verbatim}

\hypertarget{wikipedon-7}{%
\section{Wikipedon}\label{wikipedon-7}}

Cooperar com pares na produção de conteúdo digital (20 min)

A \href{https://pt.wikipedia.org/wiki/Produ\%C3\%A7\%C3\%A3o_colaborativa}{produção colaborativa} do conhecimento sobre os recursos do solo e da terra é uma das maneiras mais eficientes de garantir sua qualidade. Nessa nova fase do Wikipedon, alguns editores deixaram de lado (temporariamente) o artigo em que estavam trabalhando para colaborar na produção do artigo que você está editando. Eles analisaram o conteúdo compilado e fizeram sugestões sobre como entendem que a qualidade do artigo pode ser melhorada. Agora você tem a tarefa de \textbf{ponderar} sobre as sugestões desses editores, \textbf{implementando} aquelas que julgar pertinente.

Ao ponderar sobre as sugestões, você precisa analisar se as mesmas respeitam as políticas da Wikipédia. A principal delas é que não devem representar o ponto de vista pessoal do editor. Sugestões pertinentes precisam vir acompanhadas de menções às fontes originais dos dados e informações, ou seja, com as respectivas referências bibliográficas formatadas de acordo com as regras de \href{https://pt.wikipedia.org/wiki/Wikip\%C3\%A9dia:Livro_de_estilo/Cite_as_fontes}{citação} da Wikipédia.

Ao final da semana, você deve apresentar um parecer sobre as sugestões dos editores, descrevendo as edições que elas permitiram fazer para melhorar a qualidade do artigo.

\hypertarget{encontro-4}{%
\section{Encontro}\label{encontro-4}}

Relações entre Recursos do Solo e da Terra (50 min)

\hypertarget{recursos-adicionais-7}{%
\section{Recursos adicionais}\label{recursos-adicionais-7}}

Material sobre página de discussão de artigos da Wikipédia

\begin{itemize}
\tightlist
\item
  \href{https://commons.wikimedia.org/w/index.php?title=File\%3AUsando_a_p\%C3\%A1gina_de_discuss\%C3\%A3o.ogv}{Usando a página de discussão}
\item
  \href{https://pt.wikipedia.org/wiki/Wikip\%C3\%A9dia:P\%C3\%A1gina_de_discuss\%C3\%A3o}{Wikipédia: Página de discussão}
\item
  \href{https://pt.wikipedia.org/wiki/Ajuda:Tutorial/Discuss\%C3\%A3o}{Ajuda:Tutorial/Discussão}
\item
  \href{https://youtu.be/8APJKPMYwRA}{Como deixar mensagem em discussão de artigo na Wikipédia}
\item
  \href{https://youtu.be/BU5Ga_Yz8ZU}{Edição básica da Wikipédia: página de discussão}
\end{itemize}

\hypertarget{semana-12}{%
\chapter{Semana 12}\label{semana-12}}

\hypertarget{duxfavidas-da-semana-10}{%
\section{Dúvidas da semana}\label{duxfavidas-da-semana-10}}

Você tem dúvidas sobre as atividades da semana anterior? Quer saber mais sobre as atividades que terá que desenvolver na semana que está iniciando? Participe de nosso encontro semanal de resolução de dúvidas. Você também pode deixar uma mensagem no Fórum de Dúvidas e Sugestões.

\begin{itemize}
\tightlist
\item
  Informações gerais

  \begin{itemize}
  \tightlist
  \item
    Local: Moodle
  \item
    Horário: 11:00--11:50
  \item
    Duração prevista: 50 min
  \item
    Gravação: não
  \end{itemize}
\end{itemize}

\hypertarget{wikipedon-8}{%
\section{Wikipedon}\label{wikipedon-8}}

A principal tarefa desta semana consiste em \textbf{identificar as relações espaciais existentes entre os recursos do solo e da terra de seu município de trabalho, reportando as mesmas na seção \emph{Solos}}. Ao reportar essas relações solo-paisagem, forneca as informações necessárias para que leitores da Wikipédia sem familiaridade com ciência do solo compreendam a razão da ocorrência dos diferentes tipos de solos (e suas características) ao longo do território do município.

Enquanto você relata as relações solo-paisagem, fique atento às exigências do Wikipedon. As seis primeiras seções (apresentação, clima, geologia, relevo, hidrografia e vegetação) devem ter aproximadamente a mesma extensão. Enquanto isso, a sétima e mais importante seção deve ter a aproximadamente a mesma extensão do conjunto das seis primeiras. Por exemplo, \textbf{se cada uma das primeiras seis seções tiver dez linhas, a seção \emph{Solos} deve ter 10 x 6 = 60 linhas}.

O cumprimento da tarefa será aferido utilizando o registro do histórico de edições da página de seu município de trabalho.

\begin{itemize}
\tightlist
\item
  Duração estimada: 40 min
\end{itemize}

\hypertarget{encontro-5}{%
\section{Encontro}\label{encontro-5}}

Relações entre Recursos do Solo e da Terra

A atividade prática da semana passada será repetida pelo fato de aquela não ter ficado gravada e disponível para os estudantes que não puderam participar.

\hypertarget{semana-13}{%
\chapter{Semana 13}\label{semana-13}}

\hypertarget{duxfavidas-da-semana-11}{%
\section{Dúvidas da semana}\label{duxfavidas-da-semana-11}}

Você tem dúvidas sobre as atividades da semana anterior? Quer saber mais sobre as atividades que terá que desenvolver na semana que está iniciando? Participe de nosso encontro semanal de resolução de dúvidas. Você também pode deixar uma mensagem no Fórum de Dúvidas e Sugestões.

\begin{itemize}
\tightlist
\item
  Informações gerais

  \begin{itemize}
  \tightlist
  \item
    Local: Moodle
  \item
    Horário: 11:00--11:50
  \item
    Duração prevista: 50 min
  \item
    Gravação: não
  \end{itemize}
\end{itemize}

\hypertarget{semana-14}{%
\chapter{Semana 14}\label{semana-14}}

\hypertarget{duxfavidas-da-semana-12}{%
\section{Dúvidas da semana}\label{duxfavidas-da-semana-12}}

\textbf{Atividade}: atendimento\\
\textbf{Local}: Moodle\\
\textbf{Data}: 2020-12-01\\
\textbf{Horário}: 11:00--11:50\\
\textbf{Duração prevista}: 50 min\\
\textbf{Gravação}: não

Você tem dúvidas sobre as atividades da semana anterior? Quer saber mais sobre as atividades que terá que desenvolver na semana que está iniciando? Participe de nosso encontro semanal de resolução de dúvidas. Você também pode deixar uma mensagem no Fórum de Dúvidas e Sugestões.

\hypertarget{wikipedon-9}{%
\section{Wikipedon}\label{wikipedon-9}}

\textbf{Atividade}: projeto\\
\textbf{Local}: Wikipédia\\
\textbf{Data}: atividade assíncrona\\
\textbf{Horário}: atividade assíncrona\\
\textbf{Duração prevista}: 30 min\\
\textbf{Gravação}: não se aplica

\hypertarget{legado-para-futuros-estudantes}{%
\section{Legado para futuros estudantes}\label{legado-para-futuros-estudantes}}

\textbf{Atividade}: fórum\\
\textbf{Local}: Moodle\\
\textbf{Data}: atividade assíncrona\\
\textbf{Horário}: atividade assíncrona\\
\textbf{Duração prevista}: 10 min\\
\textbf{Gravação}: não se aplica

Produzir informações sobre os recursos do solo de determinada região geográfica pode se revelar uma tarefa bastante difícil. Diversos fatores contribuem para que encontremos dificuldades no cumprimento dessa tarefa. Algumas dessas dificuldades são mais fáceis de contornar do que outras. Em geral, quanto mais bem informados estivermos ao iniciar uma tarefa, maiores serão as chances de termos sucesso. Por isso, é fundamental compartilhar nossas experiências com outros profissionais, sejam elas de sucesso ou de fracasso.

Nesta semana, você deve refletir sobre as dificuldades que encontrou, ao longo do período letivo, para desenvolver as atividades do Wikipedon. A partir de suas reflexões, \textbf{elabore uma lista com, pelo menos, três sugestões operacionais para os estudantes que participarão do projeto Wikipedon no próximo período letivo}. As sugestões devem ter como propósito auxiliar na minimização de riscos e facilitar a resolução dos problemas que esses alunos encontrarão. Suas sugestões e de seus colegas serão compiladas e compartilhadas com os estudantes da turma do próximo período letivo logo na primeira aula.\footnote{\textbf{Dica para o professor.} Esta atividade tem dois objetivos. O primeiro deles consiste em compilar sugestões do estudantes que possam servir de base realizar melhorias operacionais no Wikipedon. O segundo, e mais importante, é que os estudantes compreendam que sua participação no Wikipedon não se encerra aqui. Além das informações produzidas sobre os recursos do solo, eles deixam importante legado ao compartilhar as experiências vividas. Essas experiências servirão de base para os estudantes que darão continuidade às atividades do Wikipedon nos próximos períodos letivos. Isso deve fortalecer seu senso de responsabilidade e comprometimento no desenvolvimento de atividades acadêmicas e profissionais.}

\hypertarget{semana-15}{%
\chapter{Semana 15}\label{semana-15}}

\hypertarget{duxfavidas-da-semana-13}{%
\section{Dúvidas da semana}\label{duxfavidas-da-semana-13}}

Você tem dúvidas sobre as atividades da semana anterior? Quer saber mais sobre as atividades que terá que desenvolver na semana que está iniciando? Participe de nosso encontro semanal de resolução de dúvidas. Você também pode deixar uma mensagem no Fórum de Dúvidas e Sugestões.

\begin{itemize}
\tightlist
\item
  Informações gerais

  \begin{itemize}
  \tightlist
  \item
    Local: Moodle
  \item
    Horário: 11:00--11:50
  \item
    Duração prevista: 50 min
  \item
    Gravação: não
  \end{itemize}
\end{itemize}

  \bibliography{book.bib,packages.bib}

\end{document}
